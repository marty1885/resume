%-----------------------------------------------------------------------------------------------------------------------------------------------%
%	The MIT License (MIT)
%
%	Copyright (c) 2021 Philip Empl
%
%	Permission is hereby granted, free of charge, to any person obtaining a copy
%	of this software and associated documentation files (the "Software"), to deal
%	in the Software without restriction, including without limitation the rights
%	to use, copy, modify, merge, publish, distribute, sublicense, and/or sell
%	copies of the Software, and to permit persons to whom the Software is
%	furnished to do so, subject to the following conditions:
%	
%	THE SOFTWARE IS PROVIDED "AS IS", WITHOUT WARRANTY OF ANY KIND, EXPRESS OR
%	IMPLIED, INCLUDING BUT NOT LIMITED TO THE WARRANTIES OF MERCHANTABILITY,
%	FITNESS FOR A PARTICULAR PURPOSE AND NONINFRINGEMENT. IN NO EVENT SHALL THE
%	AUTHORS OR COPYRIGHT HOLDERS BE LIABLE FOR ANY CLAIM, DAMAGES OR OTHER
%	LIABILITY, WHETHER IN AN ACTION OF CONTRACT, TORT OR OTHERWISE, ARISING FROM,
%	OUT OF OR IN CONNECTION WITH THE SOFTWARE OR THE USE OR OTHER DEALINGS IN
%	THE SOFTWARE.
%	
%
%-----------------------------------------------------------------------------------------------------------------------------------------------%
% Original: https://www.overleaf.com/latex/templates/modern-latex-cv/qmdwjvcrcrph


%============================================================================%
%
%	DOCUMENT DEFINITION
%
%============================================================================%

\documentclass[10pt,A4,english]{article}	


%----------------------------------------------------------------------------------------
%	ENCODING
%----------------------------------------------------------------------------------------

% we use utf8 since we want to build from any machine
\usepackage[utf8]{inputenc}		
\usepackage[USenglish]{isodate}
\usepackage{fancyhdr}
\usepackage[numbers]{natbib}

%----------------------------------------------------------------------------------------
%	LOGIC
%----------------------------------------------------------------------------------------

% provides \isempty test
\usepackage{xstring, xifthen}
\usepackage{enumitem}
\usepackage[english]{babel}
\usepackage{blindtext}
\usepackage{pdfpages}
\usepackage{changepage}
\usepackage{calc}
%----------------------------------------------------------------------------------------
%	FONT BASICS
%----------------------------------------------------------------------------------------

% some tex-live fonts - choose your own

%\usepackage[defaultsans]{droidsans}
%\usepackage[default]{comfortaa}
%\usepackage{cmbright}
\usepackage[default]{raleway}
%\usepackage{fetamont}
%\usepackage[default]{gillius}
%\usepackage[light,math]{iwona}
%\usepackage[thin]{roboto} 

% set font default
\renewcommand*\familydefault{\sfdefault} 	
\usepackage[T1]{fontenc}

% more font size definitions
\usepackage{moresize}

%----------------------------------------------------------------------------------------
%	FONT AWESOME ICONS
%---------------------------------------------------------------------------------------- 

% include the fontawesome icon set
\usepackage{fontawesome}

% use to vertically center content
% credits to: http://tex.stackexchange.com/questions/7219/how-to-vertically-center-two-images-next-to-each-other
\newcommand{\vcenteredinclude}[1]{\begingroup
	\setbox0=\hbox{\includegraphics{#1}}%
	\parbox{\wd0}{\box0}\endgroup}
\newcommand{\tab}[1]{\hspace{.2\textwidth}\rlap{#1}}
% use to vertically center content
% credits to: http://tex.stackexchange.com/questions/7219/how-to-vertically-center-two-images-next-to-each-other
\newcommand*{\vcenteredhbox}[1]{\begingroup
	\setbox0=\hbox{#1}\parbox{\wd0}{\box0}\endgroup}

% icon shortcut
\newcommand{\icon}[3] { 							
	\makebox(#2, #2){\textcolor{maincol}{\csname fa#1\endcsname}}
}	


% icon with text shortcut
\newcommand{\icontext}[4]{ 						
	\vcenteredhbox{\icon{#1}{#2}{#3}}  \hspace{2pt}  \parbox{0.9\mpwidth}{\textcolor{#4}{#3}}
}

% icon with website url
\newcommand{\iconhref}[5]{ 						
	\vcenteredhbox{\icon{#1}{#2}{#5}}  \hspace{2pt} \href{#4}{\textcolor{#5}{#3}}
}

% icon with email link
\newcommand{\iconemail}[5]{ 						
	\vcenteredhbox{\icon{#1}{#2}{#5}}  \hspace{2pt} \href{mailto:#4}{\textcolor{#5}{#3}}
}

%----------------------------------------------------------------------------------------
%	PAGE LAYOUT  DEFINITIONS
%----------------------------------------------------------------------------------------

% page outer frames (debug-only)
% \usepackage{showframe}		

% we use paracol to display breakable two columns
\usepackage{paracol}
\usepackage{tikzpagenodes}
\usetikzlibrary{calc}
\usepackage{lmodern}
\usepackage{multicol}
\usepackage{lipsum}
\usepackage{atbegshi}
% define page styles using geometry
\usepackage[a4paper]{geometry}

% remove all possible margins
\geometry{top=1cm, bottom=1cm, left=1cm, right=1cm}

\usepackage{fancyhdr}
\pagestyle{empty}

% space between header and content
% \setlength{\headheight}{0pt}

% indentation is zero
\setlength{\parindent}{0mm}

%----------------------------------------------------------------------------------------
%	TABLE /ARRAY DEFINITIONS
%---------------------------------------------------------------------------------------- 

% extended aligning of tabular cells
\usepackage{array}

% custom column right-align with fixed width
% use like p{size} but via x{size}
\newcolumntype{x}[1]{%
	>{\raggedleft\hspace{0pt}}p{#1}}%


%----------------------------------------------------------------------------------------
%	GRAPHICS DEFINITIONS
%---------------------------------------------------------------------------------------- 

%for header image
\usepackage{graphicx}

% use this for floating figures
% \usepackage{wrapfig}
% \usepackage{float}
% \floatstyle{boxed} 
% \restylefloat{figure}

%for drawing graphics		
\usepackage{tikz}			
\usepackage{ragged2e}	
\usetikzlibrary{shapes, backgrounds,mindmap, trees}

%----------------------------------------------------------------------------------------
%	Color DEFINITIONS
%---------------------------------------------------------------------------------------- 
\usepackage{transparent}
\usepackage{color}

% primary color
\definecolor{maincol}{RGB}{ 64,64,64}

% accent color, secondary
% \definecolor{accentcol}{RGB}{ 250, 150, 10 }

% dark color
\definecolor{darkcol}{RGB}{ 70, 70, 70 }

% light color
\definecolor{lightcol}{RGB}{245,245,245}

\definecolor{accentcol}{RGB}{59,77,97}



% Package for links, must be the last package used
\usepackage[hidelinks]{hyperref}

% returns minipage width minus two times \fboxsep
% to keep padding included in width calculations
% can also be used for other boxes / environments
\newcommand{\mpwidth}{\linewidth-\fboxsep-\fboxsep}



%============================================================================%
%
%	CV COMMANDS
%
%============================================================================%

%----------------------------------------------------------------------------------------
%	 CV LIST
%----------------------------------------------------------------------------------------

% renders a standard latex list but abstracts away the environment definition (begin/end)
\newcommand{\cvlist}[1] {
	\begin{itemize}{#1}\end{itemize}
}

%----------------------------------------------------------------------------------------
%	 CV TEXT
%----------------------------------------------------------------------------------------

% base class to wrap any text based stuff here. Renders like a paragraph.
% Allows complex commands to be passed, too.
% param 1: *any
\newcommand{\cvtext}[1] {
	\begin{tabular*}{1\mpwidth}{p{0.98\mpwidth}}
		\parbox{1\mpwidth}{#1}
	\end{tabular*}
}
\newcommand{\cvtextsmall}[1] {
	\begin{tabular*}{0.8\mpwidth}{p{0.8\mpwidth}}
		\parbox{0.8\mpwidth}{#1}
	\end{tabular*}
}
%----------------------------------------------------------------------------------------
%	CV SECTION
%----------------------------------------------------------------------------------------

% Renders a a CV section headline with a nice underline in main color.
% param 1: section title
\newcommand{\cvsection}[1] {
	\vspace{14pt}
	\cvtext{
		\textbf{\LARGE{\textcolor{darkcol}{#1}}}\\[-4pt]
		\textcolor{accentcol}{ \rule{0.2\textwidth}{1.5pt} } \\
	}
}

\newcommand{\cvsectionsmall}[1] {
	\vspace{14pt}
	\cvtext{
		\textbf{\Large{\textcolor{darkcol}{#1}}}\\[-4pt]
		\textcolor{accentcol}{ \rule{0.2\textwidth}{1.5pt} } \\
	}
}

\newcommand{\cvheadline}[1] {
	\vspace{16pt}
	\cvtext{
		\textbf{\Huge{\textcolor{accentcol}{#1}}}\\[-4pt]
		
	}
}

\newcommand{\cvsubheadline}[1] {
	\vspace{16pt}
	\cvtext{
		\textbf{\huge{\textcolor{darkcol}{#1}}}\\[-4pt]
		
	}
}
%----------------------------------------------------------------------------------------
%	META SKILL
%----------------------------------------------------------------------------------------

% Renders a progress-bar to indicate a certain skill in percent.
% param 1: name of the skill / tech / etc.
% param 2: level (for example in years)
% param 3: percent, values range from 0 to 1
\newcommand{\cvskill}[3] {
	\begin{tabular*}{1\mpwidth}{p{0.72\mpwidth}  r}
		\textcolor{black}{\textbf{#1}} & \textcolor{maincol}{#2}\\
	\end{tabular*}%
	
	\hspace{4pt}
	\begin{tikzpicture}[scale=1,rounded corners=2pt,very thin]
		\fill [lightcol] (0,0) rectangle (1\mpwidth, 0.15);
		\fill [accentcol] (0,0) rectangle (#3\mpwidth, 0.15);
	\end{tikzpicture}%
}


%----------------------------------------------------------------------------------------
%	 CV EVENT
%----------------------------------------------------------------------------------------

% Renders a table and a paragraph (cvtext) wrapped in a parbox (to ensure minimum content
% is glued together when a pagebreak appears).
% Additional Information can be passed in text or list form (or other environments).
% the work you did
% param 1: time-frame i.e. Sep 14 - Jan 15 etc.
% param 2:	 event name (job position etc.)
% param 3: Customer, Employer, Industry
% param 4: Short description
% param 5: work done (optional)
% param 6: technologies include (optional)
% param 7: achievements (optional)
\newcommand{\cvevent}[7] {
	
	% we wrap this part in a parbox, so title and description are not separated on a pagebreak
	% if you need more control on page breaks, remove the parbox
	\parbox{\mpwidth}{
		\begin{tabular*}{1\mpwidth}{p{0.66\mpwidth}  r}
			\textcolor{black}{\textbf{#2}} & \colorbox{accentcol}{\makebox[0.32\mpwidth]{\textcolor{white}{\textbf{#1}}}} \\
			\textcolor{maincol}{#3} & \\
		\end{tabular*}\\[8pt]
		
		\ifthenelse{\isempty{#4}}{}{
			\cvtext{#4}\\
		}
	}
	\vspace{14pt}
}


%----------------------------------------------------------------------------------------
%	 CV META EVENT
%----------------------------------------------------------------------------------------

% Renders a CV event on the sidebar
% param 1: title
% param 2: subtitle (optional)
% param 3: customer, employer, etc,. (optional)
% param 4: info text (optional)
\newcommand{\cvmetaevent}[4] {
	\textcolor{maincol} { \cvtext{\textbf{\begin{flushleft}#1\end{flushleft}}}}
	
	\ifthenelse{\isempty{#2}}{}{
		\textcolor{black} {\cvtext{\textbf{#2}} }
	}
	
	\ifthenelse{\isempty{#3}}{}{
		\cvtext{{ \textcolor{maincol} {#3} }}\\
	}
	
	\cvtext{#4}\\[14pt]
}

%---------------------------------------------------------------------------------------
%	QR CODE
%----------------------------------------------------------------------------------------

% Renders a qrcode image (centered, relative to the parentwidth)
% param 1: percent width, from 0 to 1
\newcommand{\cvqrcode}[1] {
	\begin{center}
		\includegraphics[width={#1}\mpwidth]{qrcode}
	\end{center}
}


% HEADER AND FOOOTER 
%====================================
\newcommand\Header[1]{%
	\begin{tikzpicture}[remember picture,overlay]
		\fill[accentcol]
		(current page.north west) -- (current page.north east) --
		([yshift=50pt]current page.north east|-current page text area.north east) --
		([yshift=50pt,xshift=-3cm]current page.north|-current page text area.north) --
		([yshift=10pt,xshift=-5cm]current page.north|-current page text area.north) --
		([yshift=10pt]current page.north west|-current page text area.north west) -- cycle;
		\node[font=\sffamily\bfseries\color{white},anchor=west,
		xshift=0.7cm,yshift=-0.32cm] at (current page.north west)
		{\fontsize{12}{12}\selectfont {#1}};
	\end{tikzpicture}%
}

\newcommand\Footer[1]{%
	\begin{tikzpicture}[remember picture,overlay]
		\fill[lightcol]
		(current page.south east) -- (current page.south west) --
		([yshift=-80pt]current page.south east|-current page text area.south east) --
		([yshift=-80pt,xshift=-6cm]current page.south|-current page text area.south) --
		([xshift=-2.5cm,yshift=-10pt]current page.south|-current page text area.south) --	
		([yshift=-10pt]current page.south east|-current page text area.south east) -- cycle;
		\node[yshift=0.32cm,xshift=9cm] at (current page.south) {\fontsize{10}{10}\selectfont \textbf{\thepage}};
	\end{tikzpicture}%
}


%=====================================
%============================================================================%
%
%
%
%	DOCUMENT CONTENT
%
%
%
%============================================================================%
\begin{document}
	
	\columnratio{0.31}
	\setlength{\columnsep}{2.2em}
	\setlength{\columnseprule}{4pt}
	\colseprulecolor{white}
	
	
	% LEBENSLAUF HIERE
	\AtBeginShipoutFirst{\Header{CV}\Footer{1}}
	\AtBeginShipout{\AtBeginShipoutAddToBox{\Header{CV}\Footer{2}}}
	
	\newpage
	
	\colseprulecolor{lightcol}
	\columnratio{0.31}
	\setlength{\columnsep}{2.2em}
	\setlength{\columnseprule}{4pt}
	\begin{paracol}{2}
		
		
		\begin{leftcolumn}
			%---------------------------------------------------------------------------------------
			%	META IMAGE
			%----------------------------------------------------------------------------------------
			%\includegraphics[width=\linewidth]{resources/image.jpg}	%trimming relative to image size
			
			%---------------------------------------------------------------------------------------
			%	META SKILLS
			%----------------------------------------------------------------------------------------
			\fcolorbox{white}{white}{\begin{minipage}[c][1.5cm][c]{1\mpwidth}
					\LARGE{\textbf{\textcolor{maincol}{Martin Chang}}} \\[2pt]
					\normalsize{ \textcolor{maincol} {Software Developer/Team Lead} }
			\end{minipage}} \
		
			\cvsection{Education}

			\cvmetaevent
			{2016 - 2021}
			{Computer science and engineering (B. Sc.)}
			{National Sun-Yet-Sen University\newline Special Talent Program}
			{Graduation project: \glqq HTM based Agent in
				standard RL Environment and Beefing up Framework Performance\grqq.}
			%\icontext{CaretRight}{12}{XX.XX.XXXX in Los Angeles}{black}\\[6pt]
			%\icontext{CaretRight}{12}{german}{black}\\[6pt]
			%\icontext{CaretRight}{12}{unmarried}{black}\\[6pt]
			
			
			
			\cvsection{Skills}
			
			\cvskill{C++} {12+ Yrs} {1} \\[-2pt]
			
			\cvskill{Linux} {10+ Yrs} {1} \\[-2pt]
			
			\cvskill{Concurrent/HPC\newline programming} {6+ yrs.} {0.6} \\[-2pt]
				
			\cvskill{OpenCL} {4+ yrs.} {0.4} \\[-2pt]
			
			%\cvskill{Realtime rendering} {2+ yrs.} {0.32} \\[-2pt]
			
			\cvskill{(CERN ROOT)\newline Big Data analysis} {2+ yrs.} {0.32} \\[-2pt]
			
			%\cvskill{Distributed legder \newline technology} {2+ yrs.} {0.32} \\[-2pt]
			
			%\cvskill{Internet of things} {1+ yrs.} {0.16} \\[-2pt]
			
			%\cvskill{Cloud computing} {1+ yrs.} {0.16} \\[-2pt] \\
			
			Language skills\\
			
			\cvskill{Chinese} {Native} {1} \\[-2pt]
			
			\cvskill{English (TOEIC)} {815} {0.822} \\[-2pt]
			
			\cvskill{Esperanto (Est. KER)} {C1} {0.13} \\[-2pt]
			
			\cvsection{Additional skills}

			\icontext{CaretRight}{12}{Lock free programming}{black}\\[6pt]			
			%\icontext{CaretRight}{12}{Offline rendering, PBR}{black}\\[6pt]
			\icontext{CaretRight}{12}{Lua, Python, Common Lisp}{black}\\[6pt]
			\icontext{CaretRight}{12}{Qt, PostgreSQL, GNUnet}{black}\\[6pt]
			\icontext{CaretRight}{12}{Unity, Blender, VRChat SDK}{black}\\[6pt]

			%\cvsection{Contact}
			
			%\icontext{MapMarker}{16}{Street name XX\\D-XXXXX Lorem}{black}\\[6pt]
			%\icontext{MobilePhone}{16}{+XX XXX XXXX XXXX}{black}\\[6pt]
			%\iconemail{Envelope}{16}{XXXX@XXXXX.XX}{XXXX@XXXXX.XX}{black}\\[6pt]
			%\iconhref{Home}{16}{XXX.XXX-XXXX.XX}{http://XXX-XXX.XXX.XX}{black}\\[6pt]
			\iconhref{Github}{16}{github.com/marty1885}{https://github.com/marty1885}{black}\\[6pt]
			\iconhref{Globe}{16}{clehaxze.tw}{https://clehaxze.tw/blog}{black}\\[6pt]
			%\iconhref{Xing}{16}{xing.com/user\_name}{https://www.xing.com/profile/User_Name}{black}\\
			
			\vspace*{\fill}
			Compiled \today\xspace with \LaTeX
			
		\end{leftcolumn}
		\begin{rightcolumn}
			%---------------------------------------------------------------------------------------
			%	TITLE  HEADER
			%----------------------------------------------------------------------------------------
			
			
			%---------------------------------------------------------------------------------------
			%	PROFILE
			%----------------------------------------------------------------------------------------
			\cvsection{Profile}
			\vspace{4pt}
			
			\cvtext{
				I started coding at a very young age. With \pgfmathparse{int(\year - 2010)}\pgfmathresult+ years of C++ experience. Now following the latest C++ standard and TS. I'm interested in HPC and systems programming. I am also an regular open source contributor. And an active maintainer of Drogon, one of the fastest web application framework. Currently learning the internals Unity and VR development. I'm confident in the diversity of my skill set and my ability to understand then fix freestanding bugs that I find.
			}
			
			
			%---------------------------------------------------------------------------------------
			%	WORK EXPERIENCE
			%----------------------------------------------------------------------------------------
			
			\vspace{10pt}
			\cvsection{Work experience}
			\vspace{4pt}
			
			\cvevent
			{11/2023 - CURRENT}
			{Senior Software Engineer}
			{NVIDIA Corp}
			{Developes software for the Omniverse project. Mostly in C++ and Python, integrating AI and non-conventional technology into the simulation pipeline. }
			\vfill\null
			
			\cvevent
			{01/2022 - 11/2023}
			{Native Application Lead}
			{Lumina Industries, Inc}
			{Manages and leads our software team while also contributing to the codebase. As well as handling deployment to customer environments. Does pilot implementations to reduce workload on our engineering team.}
			\vfill\null
			
			\cvevent
			{09/2021 - 01/2022}
			{C++/GPU Developer}
			{Lumina Industries, Inc}
			{Design high performance architecture for our core product, integrating modern C++17/20 technology our codebase. Introduces the team to modern tools like Sanitizers, FBInfer and cppcheck.}
			\vfill\null
			
			%\cvevent
			%{01/2016 - 08-2021}
			%{Research Assistant}
			%{Digital IP Lab, Department of CS, NSYSU}
			%{Leading the tech stack and does a variety of tasks. Including managing research server/cluster, optimize existing algorithms, apply AI to problems at hand. I also %introduced the lab to tools like Address Sanitizer, C++14/17 as well as big data %tools like ROOT. And moved all embedded development to Ubuntu/Arch Linux.}
			%\vfill\null
			
			
			\cvsection{Open source}
			\newline
			\cvtext {
				Some especially notable projects that I am involved and did major contributions.
			}
			\vspace{4pt}
				\cvevent
				{MAINTAINER}
				{Drogon - drogonframework/drogon}
				{C++14/17/20 web application framework}
				{Drogon is one of the fastest web application framework. I maintain and contributed core subsystems for the framework. Such as C++20 coroutine, HaikuOS support, Gemini protocol integration and the DrogonTest async test framework. I aslo maintain trantor - our transport library. Overhauled the TLS support and CSPRNG subsystem. Working on HTTP/2 support.}
				\vfill\null
				
				\cvevent
				{AUTHOR}
				{Etaler - etaler/Etaler}
				{High performance HTM/tensor library}
				{Etaler is a high performance implementation of Hierarchical Temporal Memory, a biologically inspired AI model by Numenta. At time of release, Etaler is more than 20x faster than the community developed HTM.core on CPU and 40x faster on a GPU. I'm also a trusted member of the HTM community.}
				\vfill\null
				
				\cvevent
				{AUTHOR}
				{TLGS - marty1885/tlgs}
				{Open source search engine for the Gemini protocol}
				{An open source search engine for contents that is served over the Gemini protocol. Written the state of the art C++20 and uses the SALSA ranking algorithm. It is now one of the 3 major search engine in Gemini space. It can also serve content over the Spartan protocol.}
				\vfill\null
			% hofixes to create fake-space to ensure the whole height is used
			%\mbox{}
			%\vfill
			
			
			%\today     \hspace{1cm}   \hrulefill
			%\hspace*{30mm}\phantom{Lorem, \today }Martin Chang
			
		\end{rightcolumn}
	\end{paracol}
	
	
\end{document}